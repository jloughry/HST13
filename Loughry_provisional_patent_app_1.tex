\documentclass[12pt,letterpaper]{article}
\usepackage{graphicx}
\usepackage{url}
\title{Provisional Application for Patent}
\author{Joe Loughry}
\date{\today}
\bibliographystyle{plain}
\begin{document}
\maketitle

\section{Title} METHOD FOR CONTROLLING SCHEDULE AND COST OF SECURITY CERTIFICATION TESTING

\section{Inventor} JOE LOUGHRY

\section{Background of the Invention}

This invention pertains to security certification testing of computer systems, including but
not limited to cross domain solutions used to construct cross domain systems.\footnote{The
terms `cross domain solution', `cross domain system', and `certification' are defined in
references \cite{Loughry2010a,Loughry2012a,Loughry2012b}.}  Although the background articles
included with this
provisional patent application and incorporated herein by reference refer to cross domain
systems processing classified information, it should be understood that the present
invention is applicable more widely, for example to systems handling privacy sensitive electronic
health care records,
to systems responsible for handling personally identifiable information, law-enforcement-sensitive
information systems, financial trading systems and record keeping systems or other types of
financial information, and to any other category of sensitive information where cross domain
systems or cross domain solutions in general are used to transfer information in a controlled manner
across security boundaries, and where security testing is done on such
systems to validate that they are fit for purpose.

The method disclosed here allows for the first time the cross domain solution developer some
measure of control over the schedule---and thereby the cost---of security certification testing.
Such testing is commonly performed by certification authorities, oftentimes governments,
on newly developed or significantly updated versions of cross domain solutions.  Security
certification testing is done on the first version of these hardware and/or software
products before they are allowed to be used in the field, and repeated whenever the
developer releases a new version or significantly changes an old version, such as would
be the case after a port to a new operating system, or different hardware, or the inclusion
of new functionality.  Certification and re-certification have historically been expensive,
both in terms of money and time.  Certification is sometimes referred to by the abbreviation
CT\&E, for Certification Test and Evaluation.  Any improvement to the CT\&E process that speeds
the schedule or reduces the total cost is potentially valuable to developers, installers, end-users,
certifiers, testers, and in the case of U.S.\ government systems, indirectly to the taxpayers.

The method disclosed here was discovered in the course of research into the reasons why
cross domain certification and accreditation are so expensive.  It has not been published,
sold, or offered for sale or public use before the date of this application.

\section{Brief Summary of the Invention}

Using grounded theory methodology, several case studies of successful and unsuccessful
security certification testing activities were analysed.  A theory was discovered, grounded
in the data, that says there is an observed correlation between reports of findings issued
by the certification authority's penetration testers and the cross domain solution software
or hardware developer's response to these finding reports.  Briefly, every time the developer
disputes the penetration testers'
findings, it tends to prompt the issuance of another report containing more findings, thereby
delaying completion of the certification process and increasing its cost, whether measured
in terms of time or money.  By using this discovery, the developer has an opportunity to
control the number of findings reported during penetration testing, and
hence at least partly control the schedule of certification testing, which directly affects
its total cost.  Expressed in the simplest possible terms, at the time that the developer wants
to terminate the penetration testing phase of the certification activity, the developer should
stop disputing the findings of the penetration testers.

Penetration testing shares several characteristics related to open-ended level of effort with
covert channel analysis \cite{NCSC-TG-030}.  Given unlimited time and funding, penetration
testers would be able to generate effectively unlimited numbers of findings.  In practice,
discrete findings
generated by penetration testers during certification testing of a particular product are limited
in part by the funding made available to the penetration testers by the certification authority.
Before the discovery described in this invention disclosure, the developer had no way to
control the schedule of this phase of the certification activity.  The developer has to this
point always been completely
at the mercy of the certification authority who controlled entirely the duration of the
certification process, and hence the cost in money and time before the product being tested
is certified and can be sold by the developer and used.  This invention balances the equation
somewhat by giving the developer a share of control over the schedule of certification testing.

\section{Brief Description of the Figures}

Figure 1 shows a flowchart of the relevant portion of the certification test activity.  After
the certification activity 1 begins and the penetration testers 2 issue their first report of
findings, the developer 3 has a choice.  If the developer 3 chooses 5 to dispute the report of
findings by the penetration testers 2, then penetration testers 2 will be prompted to issue
another report of findings and the cost of the certification activity 1--4 correspondingly
increases; if the developer 3 chooses 6 not to dispute the report of findings, then the
penetration testers 2 will be satisfied that their work has been adequately done, and the
certification activity 4 can proceed to the
end, without increasing the total cost of the certification activity 1--4 further.  Note that
there may be several `YES' iterations 5 before the `NO' branch 6 is finally taken.

\section{Detailed Description of the Invention}

The developer exerts control over the schedule of the certification process by modulating the
developer's responses to reports of findings
issued by the penetration testers.  Examples of a cross domain solution developer doing
this successfully near the end of a CT\&E activity are shown in \cite[chapter 5]{Loughry2012b}.
Counterexamples illustrating what happens when the developer did not follow the method described
in this invention, e.g., in \cite[chapter 4]{Loughry2012b} are also helpful in understanding
the utility of this invention.  It is believed that neither the developer nor the certification
authority in the case studies realised that such a control method exists or that the developer was
ineffectively using it, but the data support the theory that such a control method does exist
and is effective if used properly.  It works like this: if the developer responds by disputing
most or all of the penetration testers' findings in the $n$th findings report, then the probability
is high that the penetration testers will respond to the developer's response with a new report
containing additional findings, 
unless funding for the penetration testing effort is terminated by the
certification authority first.  If, on the other hand, the developer responds to finding
report $n$ by agreeing with the penetration testers about the reality and importance of the
findings, suggesting ways to mitigate the findings and in general concurring with the findings,
then at least in the data studied, the penetration testers were less likely to issue another
report with more findings, and the certification activity proceeded to a successful conclusion.

The reason for the existence of this correlation is believed to emerge from the nature of
penetration testing.  Like covert channel analysis---a related form of security testing
sometimes done on computer systems---penetration testing tends to be open ended
\cite{NCSC-TG-030,Pulugurtha2008}.  In contrast to regression testing and
independent verification and validation testing, in which testing begins and ends with a set of
requirements, specifications, and test procedures, penetration testing has no defined end
state; it simply continues until the penetration testers are satisfied that they have adequately
tested the device or system under test, or until the penetration testers are
told to stop by the certification authority.  What was previously unrecognised, however, is
that the cross domain solution developer in fact has some implicit input into the penetration
testers' internal decision process to continue or stop looking for more flaws in the system
or device under test.

It might be argued that this is simply human nature and it is obvious that the developer
could manipulate the penetration testers in the way described.  Evidently it is not,
however, because in the case studies described here \cite[Chapters 4--5]{Loughry2012b} both
the developer and certifier appeared to be unaware of the effect and it was only after analysis
of the project management records from the case studies that the correlation between developer
responses to
security testers' findings reports and issuance of new reports containing additional findings
was discovered.  Consider also a parallel situation in the history of software engineering
research: IBM's research in the nineteen-seventies into programmer productivity as a function
of the amount of desk space, linear feet of shelf space, the provision of interactive terminals,
deliberate absence of telephone interruptions, and private offices vs cubicle working environment,
which was prompted by observations noted by Brooks years earlier during the OS/360 project
\cite{Brooks1995}.  The results of that research were so surprising
that IBM built an entire building, the Santa Teresa Laboratory, to further test them
\cite{DeMarco1999,McCue1978}.  An existence proof independently validating the aforementioned
result can be seen in the example of Microsoft, who house their programmers in similarly
private offices and have achieved obvious success in doing so.

\newpage
\section*{Claims}

\begin{enumerate}
	\item A method usable by the developer of a cross domain solution to exert some control
		over the duration of the penetration testing phase of a security certification of
		said cross domain solution, hence over the schedule and duration of the security
		certification activity, thereby reducing cost in terms of time and/or money of
		the certification process.
\end{enumerate}

\newpage
\bibliography{consolidated_bibtex_file}

\vfill\noindent
{\tiny \LaTeX\ build \input{build_counter_1.txt}}

\newpage

\section*{Abstract}

A method for controlling the schedule and thereby the cost of security certification testing
of cross domain solutions.  The cross domain solution developer controls the number of rounds
of penetration testing by modulating the developer's formal responses to findings reports
issued by the penetration testers.  When the developer wishes to terminate the penetration
testing phase of the certification activity, the developer stops disputing the findings of
the penetration testers.

\newpage

\includegraphics{./Loughry_provisional_application_1_drawing_1.pdf}

\end{document}

