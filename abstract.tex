From direct observation of the certification (post--software-development) and accreditation
(pre-deployment) testing of cross domain systems used for the interconnection of classified
security domains in U.S.\ and U.K.\ defence and intelligence community systems, certain
characteristic
behavioural patterns have been noted.  The savvy developer can use these to exert a measure of
control over the duration and cost of certification testing and to predict the likely direction
and magnitude of the residual risk calculation performed by security accreditors in multi-lateral,
multi-level, collateral, and
compartmented site accreditations.  DCID 6/3, Common Criteria, DIACAP, and ICD 503 testing
efforts across the evolution of a long-lived software development programme were examined using
grounded theory methodology.  While discovered through investigation of classified cross domain
system testing inefficiencies, 
it is believed that the principles are applicable more widely to privacy-sensitive areas such as
electronic health care, financial, and law enforcement record keeping systems.
The first thing found was a syndrome of pathological regressive interactions amongst software
developers, managers, independent verification and validation contractors, penetration testers,
and certification authorities that resulted in schedule slippage during the certification testing
phase and, in the accreditation phase, ineffective duplication of testing with no corresponding
improvement in residual risk.  To understand why these problems occurred, an abstract model of
how security accreditors discover and agree upon the true level of residual risk in multi-level
cross domain system installations was developed.  The model is powerful enough to handle
collateral, SCI, and international cross domain systems with any number of endpoints.  It works
by establishing the visibility of threats, vulnerabilities, and mitigations from each data
owner's perspective according to the associated accreditor's clearance over the space of all
possible multi-level configurations, then identifying the smallest set of covert-channel--like
information flows necessary to reach a concord about residual risk without violating the global
security policy.
Conventional wisdom holds that security rules should be strictly enforced, but it can be shown
that under present regulations, some desirable information flows are inhibited and other
undesirable information flows are forced.  Paradoxically, it is sometimes the case that relaxing
the rules actually improves security.

