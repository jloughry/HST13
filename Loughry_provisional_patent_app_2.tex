\documentclass[12pt,letterpaper]{article}
\usepackage{graphicx}
\usepackage{url}
\title{Provisional Application for United States Patent}
\author{Joe Loughry}
\date{\today}
\bibliographystyle{plain}
\begin{document}
\maketitle

\section{Title} SYSTEM AND METHOD FOR PREDICTING THE BEHAVIOUR OF ACCREDITORS

\section{Inventor} JOE LOUGHRY

\section{Background of the Invention}


Notes: The conventional wisdom holds that security rules ought to be strictly enforced.


This invention pertains to security accreditation testing of computer systems, including
but not limited to cross domain solutions used to construct cross domain systems.\footnote{The
terms `cross domain solution', `cross domain system', and `accreditation' are defined
in references \cite{Loughry2010a,Loughry2012a,Loughry2012b}.}  Although the
scientific papers included with this provisional patent application refer to cross domain
systems processing classified information, it should be understood that the present invention
is applicable more widely, for example to privacy sensitive health care records, law
enforcement sensitive information, financial trading records and other types of financial
information, or any other category of sensitive information where cross domain systems or
cross domain solutions are used to transfer information in a controlled manner across
security boundaries, and security testing is done on cross domain solutions and cross
domain systems to validate that they are fit for purpose.

The system and method disclosed here allow the cross domain solution developer or installer
or cross domain system vendor or system integrator to predict the actions and behaviour of
accreditors during the security accreditation activity.  Accreditation is a security testing
process that tests a particular computer system that is installed in a particular location.
It is distinct from the security certification of a product, which is usually done first.
Accreditation involves an accreditor, who is a person, usually a government official,
designated to personally accept formal responsibility for the correct operation of the
system to be accredited.  Accreditation usually lasts for a defined length of time, often a
maximum of three years, and is periodically reviewed; re-accreditation may be required at
the accreditor's discretion any time a significant change is made to the accredited system
or its environment.  Many accreditors work with professional security testers who perform
security testing and advise the accreditor.  The types of security testing done on systems
undergoing accreditation often include penetration testing and Independent Verification and
Validation (IV\&V) testing.

Accreditation can seem an opaque process to developers, vendors, and installers of cross
domain solutions, system integrators who design cross domain systems incorporating cross
domain solutions, data owners responsible for systems and networks connected to cross domain
systems, and government programme offices responsible for sponsoring cross domain solution
developers.  Especially in the case of cross domain systems, which by definition always
span security boundaries controlled by mutually distrustful data owners, the cost of
accreditation is both relatively and in absolute terms expensive, whether measured in time
or money.  Accreditation of cross domain systems has been shown to be unnecessarily
expensive because of duplication of security testing caused by conflation of the principle
of defence in depth with the practice of IV\&V \cite{Loughry2010a}.  The duplication of
effort has been shown to be a consequence of inadequate communication between or amongst
accreditors representing data owners at different security classification levels and having
different security clearances \cite{Loughry2012a}.

The method disclosed here was discovered in the course of research into the reasons why
cross domain certification and accreditation are so expensive.  It has not been published,
sold, or offered for sale or public use before the date of this application.

\section{Brief Summary of the Invention}

Using grounded theory methodology, several case studies of cross domain system accreditations
were analysed.  A theory was discovered, grounded in the data, that ... 




there is an observed correlation between reports of findings issued
by the certification authority's penetration testers and the cross domain solution developer's
response to these finding reports.  Briefly, if the developer disputes the penetration testers'
findings, this tends to prompt the issuance of another report containing more findings, thereby
delaying completion of the certification process and increasing its cost, whether measured
in terms of time or money.  By using this discovery, the developer has an opportunity to
control the number of findings reported during penetration testing, and
hence at least partly control the schedule of certification testing, which directly affects
its total cost.  Expressed in the simplest possible terms, at the time that the developer wants
to terminate the penetration testing phase of the certification activity, the developer should
stop disputing the findings of the penetration testers.

Given unlimited time and funding, penetration
testers tend to generate unlimited numbers of findings.  In practice, the amount of findings
generated by penetration testers during certification testing of a particular product is limited
in part by the funding made available to the penetration testers by the certification authority.
Before the discovery described in this invention disclosure, the developer had no way to
control the schedule of certification.  The developer has to this point always been completely
at the mercy of the certification authority who controlled entirely the duration of the
certification process, and hence the cost in money and time before the product being tested
is certified and can be sold by the developer and used.  This invention balances the equation
somewhat by giving the developer a share of control over the schedule of certification testing.

\section{Brief Description of the Figures}

Figure 1 shows a flowchart of the relevant portion of the certification test activity.  After
the certification activity 1 begins and the penetration testers 2 issue their first report of
findings, the developer 3 has a choice.  If the developer 3 chooses 5 to dispute the report of
findings by the penetration testers 2, then penetration testers 2 will be prompted to issue
another report of findings and the cost of the certification activity 1--4 correspondingly
increases; if the developer 3 chooses 6 not to dispute the report of findings, then the
penetration testers 2 will be satisfied, and the certification activity 4 can proceed to the
end, without increasing the total cost of the certification activity 1--4 further.  Note that
there may be several `YES' iterations 5 before the `NO' branch 6 is finally taken.

\section{Detailed Description of the Invention}

The developer exerts control by modulating the developer's responses to reports of findings
issued by the penetration testers.  Examples of a cross domain solution developer doing
this successfully near the end of a CT\&E activity are shown in \cite[Chapter 4]{Loughry2012b}.
Counterexamples showing what happens when the developer did not follow the method described
in this invention, e.g., in \cite[Chapter 5]{Loughry2012b} are also helpful in understanding
the utility of this invention.  It is believed that the developer in the case studies never
realised that such a control method exists or that the developer was
ineffectively using it, but the data support the theory that such a control method does exist
and is effective if used properly.  It works like this: if the developer responds by disputing
most or all of the penetration testers' findings in finding report $n$, then the probability
is high that the penetration testers will respond to the developer's response with finding
report $n+1$, unless funding for the penetration testing effort is terminated by the
certification authority first.  If, on the other hand, the developer responds to finding
report $n$ by agreeing with the penetration testers about the reality and importance of the
findings, suggesting ways to mitigate the findings and in general concurring with the findings,
then at least in the data studied, the penetration testers are less likely to issue another
report with more findings, and the certification activity proceeds to a successful conclusion.

The reason for the existence of this correlation is believed to emerge from the nature of
penetration testing.  Like covert channel analysis---a related form of security testing
sometimes done on computer systems---penetration testing tends to be open ended
\cite{NCSC-TG-030,Pulugurtha2008}.  In contrast to regression testing and
independent verification and validation testing, where testing begins and ends with a set of
requirements, specifications, and test procedures, penetration testing has no defined end
state; it simply continues until the penetration testers are satisfied that they have adequately
tested the device or system under test, or until the penetration testers are
told to stop by the certification authority.  What was previously unrecognised, however, is
that the cross domain solution developer in fact has some implicit input into the penetration
testers' internal decision process to continue or stop.

It might be argued that this is simply human nature and it is obvious that the developer
could manipulate the penetration testers in the way described.  Apparently it is not obvious,
however, because in the case studies described here \cite[Chapters 4--5]{Loughry2012b} the
developer appeared to be unaware of the effect and it was only after analysis of the data
in the course of writing \emph{ibid.}\ that the correlation between developer responses to
security testers' findings reports and issuance of new reports containing additional findings
was discovered.  Consider also a parallel situation: the results of the IBM Corporation's
research in the nineteen-seventies into programmer productivity as a function of the amount
of desk space, linear feet of shelf space, presence of interactive terminal, absence of
telephone interruptions, and private offices vs cubicle working environment, which was prompted
by observations noted by Brooks during the OS/360 project \cite{Brooks1995}.  The results of
that research were so surprising
that IBM built an entire building, the Santa Teresa Laboratory, to further test them
\cite{DeMarco1999,McCue1978}.  An existence proof independently validating the aforementioned
result can be seen in the example of Microsoft, who house their programmers in similarly
private offices and have achieved obvious success in doing so.

\newpage
\section*{Claims}

\begin{enumerate}
	\item A method usable by the developer of a cross domain solution to exert some control
		over the duration of the penetration testing phase of a security certification of
		said cross domain solution, hence over the schedule and duration of the security
		certification activity, thereby reducing cost in terms of time and/or money of
		the certification process.
\end{enumerate}

\newpage
\bibliography{consolidated_bibtex_file}

\vfill\noindent
{\tiny \LaTeX\ build \input{build_counter_1.txt}}

\newpage

\section*{Abstract}

A method for controlling the schedule and thereby the cost of security certification testing
of cross domain solutions.  The cross domain solution developer controls the number of rounds
of penetration testing by modulating the developer's formal responses to findings reports
issued by the penetration testers.  When the developer wishes to terminate the penetration
testing phase of the certification activity, the developer stops disputing the findings of
the penetration testers.

\newpage

\includegraphics{./Loughry_provisional_application_1_drawing_1.pdf}

\end{document}

